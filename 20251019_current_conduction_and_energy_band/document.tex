\documentclass[12pt]{article}
\usepackage[a4paper, margin=1in]{geometry}
\usepackage{setspace}
\usepackage{titlesec}
\usepackage{amsmath, amssymb}
\usepackage{graphicx}
\usepackage{hyperref}
\setstretch{1.15}
\titleformat{\section}{\large\bfseries}{\thesection.}{0.5em}{}

\begin{document}
	
	\begin{center}
		{\LARGE \textbf{Carrier Conduction in n-Type and p-Type Semiconductors}}\\[0.5em]
		{\large Summary Notes for Semiconductor Physics}
	\end{center}
	\vspace{1em}
	
	\section{Overview}
	Conduction in semiconductors arises from the movement of mobile charge carriers under an applied electric field:
	\begin{itemize}
		\item \textbf{n-type:} Majority carriers are electrons in the conduction band.
		\item \textbf{p-type:} Majority carriers are holes in the valence band.
	\end{itemize}
	Although both mechanisms produce the same macroscopic effect---electric current---their microscopic details differ.
	
	\section{n-Type Conduction Mechanism}
	\begin{itemize}
		\item Donor atoms contribute free electrons to the \textbf{conduction band}.
		\item These conduction-band electrons are \textbf{delocalized} and respond directly to the electric field.
		\item When an electric field is applied, electrons drift opposite to the field direction, while current (conventional flow) is in the field direction.
		\item The electrons remain entirely in the conduction band as they move; they do \emph{not} need to recombine with holes.
		\item Recombination is negligible in n-type material because holes are extremely scarce.
	\end{itemize}
	
	\section{p-Type Conduction Mechanism}
	\begin{itemize}
		\item Acceptor atoms create empty states (holes) near the top of the \textbf{valence band}.
		\item The valence band is almost full, but a small number of electrons can move between neighboring atoms by filling nearby holes.
		\item Each electron hop fills a hole and creates a new one in its previous location---the holes thus appear to drift in the direction of the electric field.
		\item Electrons in p-type conduction remain \textbf{within the valence band} at all times; they never cross into the conduction band.
		\item At the negative terminal, electrons from the external circuit enter the semiconductor and recombine with holes near that contact.
		\item At the positive terminal, valence electrons are withdrawn, creating new holes to sustain the current flow.
	\end{itemize}
	
	\section{Comparison of n-Type and p-Type Conduction}
	\begin{table}[h!]
		\centering
		\renewcommand{\arraystretch}{1.2}
		\begin{tabular}{|c|c|c|c|c|}
			\hline
			\textbf{Type} & \textbf{Majority Carriers} & \textbf{Band of Motion} & \textbf{Carrier Drift Direction} & \textbf{Contact Interaction} \\
			\hline
			n-type & Electrons & Conduction band & Opposite to electric field & Electrons supplied/removed via conduction band \\
			\hline
			p-type & Holes (valence electrons hopping) & Valence band & Same as electric field & Electrons recombine or are extracted at contacts \\
			\hline
		\end{tabular}
		\caption{Comparison of carrier motion and contact behavior in n-type and p-type semiconductors.}
	\end{table}
	
	\section{Conceptual Summary}
	\begin{itemize}
		\item In \textbf{n-type materials}, current is carried by electrons freely moving in the conduction band.
		\item In \textbf{p-type materials}, current is carried by holes moving in the valence band---realized physically as valence electrons hopping between holes.
		\item The external circuit continuously supplies and removes electrons at the contacts to maintain steady-state carrier motion.
	\end{itemize}
	
	\noindent
	\textbf{In one sentence:}  
	Electrons in n-type semiconductors flow freely through the conduction band, while in p-type materials they remain in the valence band, hopping between holes whose drift constitutes the observed current.\\\\
	n-type semicondcutor : newly supplied electrons from the negative side of the field always stays in conduction band\\
	p-type semiconductor : newly supplied electrons from the negative side of the field hops between holes, which are in the valence band. so all the moving electrons are always in the valence band.
	
\end{document}