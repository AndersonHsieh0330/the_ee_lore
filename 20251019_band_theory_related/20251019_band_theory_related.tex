\documentclass[12pt]{article}
\usepackage[a4paper, margin=1in]{geometry}
\usepackage{setspace}
\usepackage{titlesec}
\usepackage{amsmath, amssymb}
\usepackage{graphicx}
\usepackage{hyperref}

\setstretch{1.15}
\titleformat{\section}{\large\bfseries}{\thesection.}{0.5em}{}
\titleformat{\subsection}{\normalsize\bfseries}{\thesubsection.}{0.5em}{}

\begin{document}
	
	\begin{center}
		{\LARGE \textbf{Bandgap Formation and Atomic Origin: Summary Notes}}\\[0.5em]
		{\large Compiled for Semiconductor Physics Review}
	\end{center}
	\vspace{1em}
	
	\section{From Atoms to Bands}
	In isolated atoms, electrons occupy discrete energy levels (orbitals). When many atoms come together to form a solid, those discrete levels \textbf{split into bands} due to the Pauli exclusion principle.
	
	\begin{itemize}
		\item The \textbf{valence band} arises from the bonding combinations of the outer (valence) orbitals.
		\item The \textbf{conduction band} arises from the antibonding combinations of those same orbitals.
		\item The \textbf{bandgap} ($E_g$) is the energy separation between these two bands.
	\end{itemize}
	
	\section{Which Orbitals Form the Bands}
	Only the \textbf{outermost (valence) electrons} contribute meaningfully to bonding and band formation.
	
	\begin{itemize}
		\item Inner (core) electrons form deep, narrow ``core bands'' that remain inert and do not affect conduction.
		\item Example materials:
		\begin{itemize}
			\item Silicon: valence orbitals $3s$, $3p \rightarrow sp^3$ hybrids $\Rightarrow E_g \approx 1.1\,\text{eV}$.
			\item Germanium: valence orbitals $4s$, $4p \rightarrow sp^3$ hybrids $\Rightarrow E_g \approx 0.67\,\text{eV}$.
		\end{itemize}
	\end{itemize}
	
	\section{Why Bandgaps Differ Between Materials}
	Different elements have valence electrons in different shells (different principal quantum numbers $n$).
	
	\begin{itemize}
		\item Higher $n \Rightarrow$ orbitals are \textbf{larger and more diffuse}, lying farther from the nucleus.
		\item Larger orbitals $\Rightarrow$ \textbf{weaker overlap} between neighboring atoms.
		\item Weaker overlap $\Rightarrow$ smaller bonding--antibonding energy splitting $\Rightarrow$ smaller bandgap.
	\end{itemize}
	
	\subsection*{Trend Example (Group IV Semiconductors)}
	\[
	\text{C (diamond)} \rightarrow \text{Si} \rightarrow \text{Ge} \rightarrow \text{Sn}
	\]
	\[
	E_g: \; 5.5~\text{eV} \; \rightarrow \; 1.1~\text{eV} \; \rightarrow \; 0.67~\text{eV} \; \rightarrow \; \approx 0~\text{eV}
	\]
	
	\section{About the Principal Quantum Number ($n$)}
	The principal quantum number $n$ labels how far an orbital generally sits from the nucleus (its spatial extent).
	
	\begin{itemize}
		\item Larger $n$ $\Rightarrow$ electrons are farther out, more weakly bound, and have higher energy.
		\item Within a given $n$, subshells ($s$, $p$, $d$) differ in energy due to penetration and shielding effects.
		\item In semiconductor physics, $n$ mainly identifies which orbitals form the \textbf{outermost, bonding-active shell}.
	\end{itemize}
	
	\section{Higher Energy vs. Weaker Overlap}
	\begin{itemize}
		\item Orbitals with higher atomic energy (e.g., $4s$ vs.\ $3s$) are farther from the nucleus and more diffuse.
		\item Diffuse orbitals overlap less effectively between neighboring atoms.
		\item Less overlap $\Rightarrow$ weaker bonding interaction $\Rightarrow$ smaller bonding--antibonding splitting $\Rightarrow$ smaller bandgap.
	\end{itemize}
	
	Thus, ``higher atomic energy'' and ``bonding strength'' trend in opposite directions.
	
	\section{Key Takeaway}
	The valence and conduction bands in a semiconductor come from the bonding and antibonding combinations of the \textbf{outermost valence-shell orbitals}. Higher-$n$ orbitals are larger and overlap less strongly, so the bonding--antibonding energy splitting---and therefore the bandgap---\textbf{decreases} as you move down the periodic table.
	
\end{document}
